\documentclass[11pt]{article}
\usepackage{amsmath, amsfonts, amsthm, amssymb}  % Some math symbols
\usepackage{enumerate}
\usepackage{fullpage}

\usepackage[x11names, rgb]{xcolor}
\usepackage{tikz}
\usepackage[colorlinks=true, urlcolor=blue]{hyperref}
\usepackage{graphicx}
\usetikzlibrary{snakes,arrows,shapes}

\usepackage{listings}
\usepackage{array}
\usepackage{mathtools}
\setlength{\parindent}{0pt}
\setlength{\parskip}{5pt plus 1pt}
\pagestyle{empty}

\def\indented#1{\list{}{}\item[]}
\let\indented=\endlist

\newcounter{questionCounter}
\newenvironment{question}[2][\arabic{questionCounter}]{%
    \addtocounter{questionCounter}{1}%
    \setcounter{partCounter}{0}%
    \vspace{.25in} \hrule \vspace{0.5em}%
        \noindent{\bf #2}%
    \vspace{0.8em} \hrule \vspace{.10in}%
}{}

\newcounter{partCounter}[questionCounter]
\renewenvironment{part}[1][\alph{partCounter}]{%
    \addtocounter{partCounter}{1}%
    \vspace{.10in}%
    \begin{indented}%
       {\bf (#1)} %
}{\end{indented}}

%%%%%%%%%%%%%%%%% Identifying Information %%%%%%%%%%%%%%%%%
%% This is here, so that you can make your homework look %%
%% pretty when you compile it.                           %%
%%%%%%%%%%%%%%%%%%%%%%%%%%%%%%%%%%%%%%%%%%%%%%%%%%%%%%%%%%%
\newcommand{\myhwname}{Homework 3}
\newcommand{\mysection}{CS Fundamentals: Scratch Visuals [due Tuesday]}
%%%%%%%%%%%%%%%%%%%%%%%%%%%%%%%%%%%%%%%%%%%%%%%%%%%%%%%%%%%
\begin{document}
\begin{center}
    {\Large \myhwname} \\
    \mysection \\
    \today
\end{center}

%%%%%%%%%%%%%%%%% PROBLEM 1: Probability Review %%%%%%%%%%%%%%%%%%%%%%%%%
\section{Hide the Cat! [Expected Duration: 15 - 60 min]}
\textbf{If you have any questions about the directions or any blocks you have not used before, let me know via email or text!}\\
\noindent\makebox[\linewidth]{\rule{\paperwidth}{0.4pt}}\\
Again, I would \textbf{strongly recommend installing the \href{https://scratch.mit.edu/scratch2download/}{offline version of Scratch 2}} for future projects.\\\\
In last class, we used some ``Looks'' blocks to find the cat. This time, you will build a program that hides it! Since we used a lot of the blocks in the ``Looks'' category in class, I will not give you specific directions. I will only give you specific blocks you will need for each event (There is only 2 events).I have provided a YouTube video \textbf{\href{https://youtu.be/Ctk1z6EHQrg}{(CLICK HERE!)}} to help guide you.\\
Do your best and have fun!\\
\noindent\makebox[\linewidth]{\rule{\paperwidth}{0.4pt}}\\
Here are the list of blocks you need from the ``looks'' category (\textbf{these are the only blocks you need})! You should figure out the order, and where each block goes in each event (orange block).\\\\
\textbf{Note: all of the cheat keys you made works! So, you can play around with those to test your hidden cat out!}\\
\noindent\makebox[\linewidth]{\rule{\paperwidth}{0.4pt}}
\begin{enumerate}
\item When Flag Clicked
\begin{enumerate}[a.]
\item go back $\rule{0.5cm}{0.15mm}$ layers
\item go to front
\item say $\rule{0.5cm}{0.15mm}$ for $\rule{0.5cm}{0.15mm}$ secs
\item set size to $\rule{0.5cm}{0.15mm}$\%
\item show
\end{enumerate}
\end{enumerate}
\noindent\makebox[\linewidth]{\rule{\paperwidth}{0.4pt}}
\begin{enumerate}
\item When this sprite clicked
\begin{enumerate}[a.]
\item go back $\rule{0.5cm}{0.15mm}$ layers
\item go to front
\item hide
\item say $\rule{0.5cm}{0.15mm}$ for $\rule{0.5cm}{0.15mm}$ secs
\item set size to $\rule{0.5cm}{0.15mm}$\%
\item think $\rule{0.5cm}{0.15mm}$ for $\rule{0.5cm}{0.15mm}$ secs
\end{enumerate}
\end{enumerate}
\end{document}
