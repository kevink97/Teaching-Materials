\documentclass[11pt]{article}
\usepackage{amsmath, amsfonts, amsthm, amssymb}  % Some math symbols
\usepackage{enumerate}
\usepackage{fullpage}

\usepackage[x11names, rgb]{xcolor}
\usepackage{tikz}
\usepackage[colorlinks=true, urlcolor=blue]{hyperref}
\usepackage{graphicx}
\usepackage{gensymb}

\usetikzlibrary{snakes,arrows,shapes}

\usepackage{listings}
\usepackage{array}
\usepackage{mathtools}
\setlength{\parindent}{0pt}
\setlength{\parskip}{5pt plus 1pt}
\pagestyle{empty}

\def\indented#1{\list{}{}\item[]}
\let\indented=\endlist

\newcounter{questionCounter}
\newenvironment{question}[2][\arabic{questionCounter}]{%
    \addtocounter{questionCounter}{1}%
    \setcounter{partCounter}{0}%
    \vspace{.25in} \hrule \vspace{0.5em}%
        \noindent{\bf #2}%
    \vspace{0.8em} \hrule \vspace{.10in}%
}{}

\newcounter{partCounter}[questionCounter]
\renewenvironment{part}[1][\alph{partCounter}]{%
    \addtocounter{partCounter}{1}%
    \vspace{.10in}%
    \begin{indented}%
       {\bf (#1)} %
}{\end{indented}}

%%%%%%%%%%%%%%%%% Identifying Information %%%%%%%%%%%%%%%%%
%% This is here, so that you can make your homework look %%
%% pretty when you compile it.                           %%
%%%%%%%%%%%%%%%%%%%%%%%%%%%%%%%%%%%%%%%%%%%%%%%%%%%%%%%%%%%
\newcommand{\myhwname}{Homework 4}
\newcommand{\mysection}{Review: the Weather App [due Friday]}
%%%%%%%%%%%%%%%%%%%%%%%%%%%%%%%%%%%%%%%%%%%%%%%%%%%%%%%%%%%
\begin{document}
\begin{center}
    {\Large \myhwname} \\
    \mysection \\
    \today
\end{center}

%%%%%%%%%%%%%%%%% PROBLEM 1: Probability Review %%%%%%%%%%%%%%%%%%%%%%%%%
\noindent\makebox[\linewidth]{\rule{\paperwidth}{0.4pt}}\\
\textbf{If you have any questions, let me know via email or text!}\\\\
In the last lesson, we built together a very cool weather app together. We built this cool weather app using many things we learned... such as:\\
\begin{itemize}
    \item if, elif, else
    \item loops (while, for)
    \item in
    \item lists
\end{itemize}
In this homework, we will review and analyze the code we coded together.\\
Refer to code.pdf which was attached to the email.\\
I want you to use the \textbf{word} bank below to complete this assignment. 
Note that \textbf{the words in the word bank is used exactly once... all words are used}.\\\\
\noindent\makebox[\linewidth]{\rule{\paperwidth}{0.4pt}}\\
\begin{center}
\textbf{Word Bank:}
\begin{tabular}{|c|c|c|c|}
    \hline
    0 & 10 & 2 & 2 words \\
    \hline
    28 & 3 & 67 & Thank you for using this app \\
    \hline
    ch & continue & days & empty string \\
    \hline
    exit & gf & gt & list \\
    \hline
    loop & o & opt & option \\
    \hline
    option & prints & q & understand \\
    \hline
    valid & will & will & will \\
    \hline
    will not & will not & zipcode & zipcode \\
    \hline
    zicode is invalid & & & \\
    \hline
\end{tabular}
\end{center}
\noindent\makebox[\linewidth]{\rule{\paperwidth}{0.4pt}}
\section{Checking the User's Zippers}
 % $\rule{2.5cm}{0.15mm}$}
\begin{enumerate}
    \item In line 13, we are asking the user for their $\rule{2.5cm}{0.15mm}$.
    \item The variable that stores the user's zipcode is called $\rule{2.5cm}{0.15mm}$.
    \item The purpose of line 15 to 17 is to make certain that the user inputed a $\rule{2.5cm}{0.15mm}$ zipcode.
    \item The meaning of 
    \item In Line 15, if zipcode = ``98105'', then the program $\rule{2.5cm}{0.15mm}$ go into the body (line 16 and 17) of the if statement.
    \item In Line 15, if zipcode = ``124567334'', then the program $\rule{2.5cm}{0.15mm}$ go into the body (line 16 and 17) of the if statement.
    \item In Line 15, if zipcode = ``4321'', then the program $\rule{2.5cm}{0.15mm}$ go into the body (line 16 and 17) of the if statement (Note: this may be a bit tricky).
    \item If the program did end up going into the body of the if statement (line 16 and 17), the program will print ``$\rule{2.5cm}{0.15mm}$'' and $\rule{2.5cm}{0.15mm}$ the program.
\end{enumerate}
\noindent\makebox[\linewidth]{\rule{\paperwidth}{0.4pt}}
\section{$o$, you want to know the options?}
 % $\rule{2.5cm}{0.15mm}$}
\begin{enumerate}
    \item In Line 28, we are asking the user for an $\rule{2.5cm}{0.15mm}$.
    \item The variable that stores the user's option is called $\rule{2.5cm}{0.15mm}$.
    \item In line 29, we are checking if the user inputted $\rule{2.5cm}{0.15mm}$ which stands for $\rule{2.5cm}{0.15mm}$.
    \item In line 30, we call a function called print$\_$options() which is defined from line $\rule{2.5cm}{0.15mm}$ to $\rule{2.5cm}{0.15mm}$.
    \item print$\_$option() function $\rule{2.5cm}{0.15mm}$ out all the options available for the user.
    \item On line 31, $\rule{2.5cm}{0.15mm}$ means that we will not look at any code from line 32 to 70 and go straight back to the beginning of the loop at line $\rule{2.5cm}{0.15mm}$.
\end{enumerate}
\noindent\makebox[\linewidth]{\rule{\paperwidth}{0.4pt}}
\section{Don't be a $q$uitter!}
 % $\rule{2.5cm}{0.15mm}$}
\begin{enumerate}
    \item If the option the user typed was not $o$ in line 29, then we check if the option inputted $\rule{2.5cm}{0.15mm}$ which stands for quit.
    \item If the option the user inputted was $q$, then we print ``$\rule{2.5cm}{0.15mm}$'' in line 34.
    \item If the option the user inputted was $q$, then, in line 34, we break which means that we leave the $\rule{2.5cm}{0.15mm}$. Since there is nothing after the loop, the program ends after ``breaking''.
\end{enumerate}
\noindent\makebox[\linewidth]{\rule{\paperwidth}{0.4pt}}
\section{I d0N't kN0w Waht tH1s m3an5}
 % $\rule{2.5cm}{0.15mm}$}
\begin{enumerate}
    \item In this code, there are $\rule{2.5cm}{0.15mm}$ places where we notice that the option the user inputted was incorrect.
    \item In line 35-37, we print ``invalid query'' because the user inputted an $\rule{2.5cm}{0.15mm}$.
    \item In line 48-49, we print ``invalid query'' because the query had 1 word, but the first word was neither `gt' nor `$\rule{2.5cm}{0.15mm}$'.
    \item In line 66-67, we print ``invalid query'' because the query had $\rule{2.5cm}{0.15mm}$ and the second word was a number and the first word was neither `$\rule{2.5cm}{0.15mm}$' nor `gf'.
    \item In line 69-70, we print ``invalid query'' because the code was not able to $\rule{2.5cm}{0.15mm}$ what the user was trying to say.
    \item If option = `gt', then the program $\rule{2.5cm}{0.15mm}$ print ``invalid query''.
    \item If option = `gt 1 2', then the program $\rule{2.5cm}{0.15mm}$ print ``invalid query''.
\end{enumerate}
\noindent\makebox[\linewidth]{\rule{\paperwidth}{0.4pt}}
\section{Ok python, what's the weather like in 2 days?}
 % $\rule{2.5cm}{0.15mm}$}
\begin{enumerate}
    \item The user must type `gf $\rule{2.5cm}{0.15mm}$' to get the weather in 2 days.
    \item The code that would get this result is in line $\rule{2.5cm}{0.15mm}$.
    \item While humans start counting from 1, computers start counting from $\rule{2.5cm}{0.15mm}$.
\end{enumerate}

\noindent\makebox[\linewidth]{\rule{\paperwidth}{0.4pt}}
\section{Ok python, when is it going to rain?}
\begin{enumerate}
    \item The user must type '$\rule{2.5cm}{0.15mm}$ rain' if the user wants to get the days of when it is going to rain.
    \item From line 53 to 57, we are storing all the $\rule{2.5cm}{0.15mm}$ in numbers whose forecast contains the query that the user specified.
    \item days variable in line 53 is of type $\rule{2.5cm}{0.15mm}$.
\end{enumerate}
\end{document}