\documentclass[11pt]{article}
\usepackage{amsmath, amsfonts, amsthm, amssymb}  % Some math symbols
\usepackage{enumerate}
\usepackage{fullpage}

\usepackage[x11names, rgb]{xcolor}
\usepackage{tikz}
\usepackage[colorlinks=true, urlcolor=blue]{hyperref}
\usepackage{graphicx}
\usepackage{gensymb}

\usetikzlibrary{snakes,arrows,shapes}

\usepackage{listings}
\usepackage{array}
\usepackage{mathtools}
\setlength{\parindent}{0pt}
\setlength{\parskip}{5pt plus 1pt}
\pagestyle{empty}

\def\indented#1{\list{}{}\item[]}
\let\indented=\endlist

\newcounter{questionCounter}
\newenvironment{question}[2][\arabic{questionCounter}]{%
    \addtocounter{questionCounter}{1}%
    \setcounter{partCounter}{0}%
    \vspace{.25in} \hrule \vspace{0.5em}%
        \noindent{\bf #2}%
    \vspace{0.8em} \hrule \vspace{.10in}%
}{}

\newcounter{partCounter}[questionCounter]
\renewenvironment{part}[1][\alph{partCounter}]{%
    \addtocounter{partCounter}{1}%
    \vspace{.10in}%
    \begin{indented}%
       {\bf (#1)} %
}{\end{indented}}

%%%%%%%%%%%%%%%%% Identifying Information %%%%%%%%%%%%%%%%%
%% This is here, so that you can make your homework look %%
%% pretty when you compile it.                           %%
%%%%%%%%%%%%%%%%%%%%%%%%%%%%%%%%%%%%%%%%%%%%%%%%%%%%%%%%%%%
\newcommand{\myhwname}{Homework 5}
\newcommand{\mysection}{CS Fundamentals: Pong Review: Math! [due Tuesday]}
%%%%%%%%%%%%%%%%%%%%%%%%%%%%%%%%%%%%%%%%%%%%%%%%%%%%%%%%%%%
\begin{document}
\begin{center}
    {\Large \myhwname} \\
    \mysection \\
    \today
\end{center}

%%%%%%%%%%%%%%%%% PROBLEM 1: Probability Review %%%%%%%%%%%%%%%%%%%%%%%%%
\section{Review: Pong Game [Expected Duration: 15 - 20 min]}
\textbf{If you have any questions about the directions or any blocks you have not used before, let me know via email or text!}\\\\
\noindent\makebox[\linewidth]{\rule{\paperwidth}{0.4pt}}\\
In the last lesson, we built together a very cool pong game together. 
We built this cool calculator by using many things we learned previously and MATH. 
For homework, I want you to do some math with angles. Math is quite important in programming! Have fun!\\\\ 
I want you to use the \textbf{word} bank below to complete this assignment. 
Note that \textbf{the words in the word bank is used exactly once... all words are used}.\\\\
If you have any questions, \textbf{please} let me know!\\\\
\noindent\makebox[\linewidth]{\rule{\paperwidth}{0.4pt}}\\
\textbf{Instructions/Notes:}\\
$0\degree = 360\degree = -360\degree$ is Up.\\
$90\degree = -270 \degree$ is East.\\
$-90\degree = 270 \degree$ is West.\\
$180\degree = -180\degree$ is South.\\\\
This homework is hard! You will need to know how to do reflection over x and y-axis. Do as best as you can! It is okay to not finish. Looking at the code for the pong game can help!
\\
\noindent\makebox[\linewidth]{\rule{\paperwidth}{0.4pt}}
\begin{enumerate}
\item \textbf{Bouncing off of the top wall}\\
The only walls the ball in pong can bounce off of are the ones on the bottom and at the top. Drawing pictures may help solve the problems below.
 % $\rule{2.5cm}{0.15mm}$
\begin{enumerate}[a.]
\item Let's say a ball was approaching the top wall at a $-45\degree$ (NorthWest) angle. To simulate bouncing off of the wall, we send the ball going at a $-135\degree$ (SouthWest) angle. Was this a reflection over the x-axis or the y-axis (think 2D coordinate! if you don't know what this means, email me!)?\\\\\\
\item If the ball was approaching the top wall at a $50 \degree$ angle, what direction (degree) should the ball be traveling once the ball bounces? (Think reflection.)\\\\\\
\item Generally, if the ball was approaching the top wall at a $x \degree$ angle, what direction $y \degree$ shoudld the ball be traveling once the ball bounces? Write an equation for $y$ in terms of $x$. (note: this is hard! try your best! the code can help!)\\\\\\\\\\\\
\end{enumerate}
\end{enumerate}
\noindent\makebox[\linewidth]{\rule{\paperwidth}{0.4pt}}
\begin{enumerate}
\item \textbf{Bouncing off of the bottom wall}\\
The only walls the ball in pong can bounce off of are the ones on the bottom and at the top. Drawing pictures may help solve the problems below.
 % $\rule{2.5cm}{0.15mm}$
\begin{enumerate}[a.]
\item Let's say a ball was approaching the bottom wall at a $135\degree$ (Southeast) angle. To simulate bouncing off of the wall, we send the ball going at a $45\degree$ (Northeast) angle. Was this a reflection over the x-axis or the y-axis (think 2D coordinate! if you don't know what this means, email me!)?\\\\\\
\item If the ball was approaching the bottom wall at a $-100 \degree$ angle, what direction (degree) should the ball be traveling once the ball bounces? (Think reflection.)\\\\\\
\item Generally, if the ball was approaching the bottom wall at a $x \degree$ angle, what direction $y \degree$ shoudld the ball be traveling once the ball bounces? Write an equation for $y$ in terms of $x$. (note: this is hard! try your best! the code can help!)\\\\\\\\\\\\\\\\\\\\\\
\end{enumerate}
\end{enumerate}
\begin{enumerate}
\item \textbf{Bouncing off of the right player's paddle}\\
There are 2 players in pongs. Each has a paddle. The balls can bounce off these paddles. Drawing pictures may help solve the problems below.
 % $\rule{2.5cm}{0.15mm}$
\begin{enumerate}[a.]
\item Let's say a ball was approaching the right player's paddle  at a $135\degree$ (Southeast) angle. To simulate bouncing off of the paddle, we send the ball going at a $-135 \degree$ (Southwest) angle. Was this a reflection over the x-axis or the y-axis (think 2D coordinate! if you don't know what this means, email me!)?\\\\\\
\item If the ball was approaching the right player's paddle at a $100 \degree$ angle, what direction (degree) should the ball be traveling once the ball bounces? (Think reflection again.)\\\\\\
\item Generally, if the ball was approaching the right player's paddle at a $x \degree$ angle, what direction $y \degree$ shoudld the ball be traveling once the ball bounces? Write an equation for $y$ in terms of $x$. (note: this is hard! try your best! the code can help!)\\\\\\\\\\\\\\\\\\\\\\\\\\\\\\\\\\\\\\\\\\\\\\\\\\\\
\end{enumerate}
\end{enumerate}
\begin{enumerate}
\item \textbf{Bouncing off of the left player's paddle}\\
There are 2 players in pongs. Each has a paddle. The balls can bounce off these paddles. Drawing pictures may help solve the problems below.
 % $\rule{2.5cm}{0.15mm}$
\begin{enumerate}[a.]
\item Let's say a ball was approaching the left player's paddle  at a $-45\degree$ (NorthWest) angle. To simulate bouncing off of the paddle, we send the ball going at a $45 \degree$ (NorthEast) angle. Was this a reflection over the x-axis or the y-axis (think 2D coordinate! if you don't know what this means, email me!)?\\\\\\
\item If the ball was approaching the right player's paddle at a $120 \degree$ angle, what direction (degree) should the ball be traveling once the ball bounces? (Think reflection again.)\\\\\\
\item Generally, if the ball was approaching the right player's paddle at a $x \degree$ angle, what direction $y \degree$ shoudld the ball be traveling once the ball bounces? Write an equation for $y$ in terms of $x$. (note: this is hard! try your best! the code can help!)
\end{enumerate}
\end{enumerate}
\end{document}
